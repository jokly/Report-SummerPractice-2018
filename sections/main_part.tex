Описание задачи: необходимо было исправить некоторые ошибки, а также добавить новый функционал
в существующие проекты. Во время работы использовалась система контроля версия git и система управления 
репозиториями GitLab для коммуникации с остальными участниками команды. Для каждой задачи создавалась 
отдельная ветка, которая в дальнейшем после рассмотрения главным backend-программистом, включалась в 
основную. Весь код был написан в соответствии с принятым стандартом в компании и успешно внедрен в проекты.

\subsect{1. Экранировать символы в URL}
Данная проблема заключалась в том, что в пути к файлам на сервере содержались символы, которые 
необходимо экранировать для их корректной обработки браузером. Для этого была написана функция в 
основном модуле и добавлена в плагин, который в дальнейшем может использовать frontend-программист. 
Функция разбивает путь к файлу на директории и экранирует каждую по отдельности, далее обратно 
собирает валидный путь. \\

\subsect{2. Добавить режим «Обслуживание»}
Необходимо было добавить функционал, который бы позволял показывать информационную страницу пользователю 
во время технических работ на сайте. Для этого была использована функциональность фреймворка Laravel
(на нем написаны все проекты), которая описана в документации. Также была сверстана страница и 
реализован перевод на два языка: русский и английский.

\subsect{3. Добавить включение содержимого файла на страницу}
Frontend-программистам необходимо было предоставить функцию по включению содержимого различных типов файлов таких, 
как:
\begin{enumerate}
    \item SVG
    \item JSON
\end{enumerate}
Для этого была написана функция, которая принимала на вход путь к файлу и проверяла его расширение MIME типу. 
Валидация происходит, основываясь на массив разрешенных типов, который могут обновлять другие разработчики в 
проекте.

\subsect{4. Перенести функцию изменения размера картинок и их кэширования на другие проекты}
В одном из проектов была реализована собственная функция по изменению размера, кэшированию и оптимизации 
картинок, которую необходимо было перенести на другие проекты (10 проектов). Для этого было удалить старый 
плагин, обновить конфигурационные файлы и модули, а также добавить новую функцию, которая предоставляла бы 
интерфейс аналогично старой. После программной реализации требовалось развернуть проекты на тестовых серверах 
и проверить корректность работы.

\subsect{5. Реализовать проверку принадлежности ссылки на файл к текущему серверу}
Во многих проектах присутствуют изображения, которые добавлены по средством ссылки на них вида: \\
http://www.example.com/uploads/img/image.png. \\ Для того, чтобы сервер мог изменять размер и кэшировать изображения, 
необходимо было получить абсолютный путь к ним при том условии, что ссылка относится к текущему серверу. Согласно 
этому требованию была написана функция, которая парсила ссылку и проверяла на соответствие домен сервера. При 
успешной проверке возвращается абсолютный путь к изображению, а в обратном случае исходная ссылка. Данная 
функциональность была применена к задаче №4.
