\sect{Заключение}

В приведенном отчете была изучена работа веб-студии «FEIP». Получены знания в работе с языком 
программирования PHP и фреймворком Laravel. А также был получен опыт с пакетным менеджером Composer, 
углублены знания в работе с системой контроля версий git, получены знания по «развертке» сайта на сервере.
Практика является актуальной для прохождения в связи с получением мною профессиональных навыков, а также 
получением первичных навыков по работе в коллективе.\\

Возникшие проблемы за время прохождения практики и пути их решения:
\begin{enumerate}
    \item Задание, которое было дано, подразумевало собой дописывание существующего функционала.
    Соответственно нужно было разбираться в уже написанном коде и дорабатывать его.
    \item Для успешного прохождения практики требовалось изучить язык программирования PHP и 
    фреймворк Laravel. Для этого были прочтены две документации соответственно.
    \item Также в коде присутствовали ошибки, которые препятствовали «развертке» сайта на локальном
    компьютере под операционно системой Windows. В ходе выполнения заданий были произведены небольшие 
    изменения в проектах для устранения ошибок.
    \item В процессе написания кода требовалось придерживаться единого стиля кодирования компании. Путем 
    обсуждения написанного кода на GitLab с главным backend-программистом удалось соблюсти все требования.
    \item Во время прохождения практики требовалось углубленное знание системы контроля версий git. В ходе 
    обсуждения проблем с другими сотрудниками был получен необходимый опыт и применен на практике.
    \item Большая часть выполненного функционала подразумевает дальнейшее использование frontend-разработчиками.
    В связи с этим, во время работы было необходимо общаться с сотрудниками для успешного выполнения поставленных 
    задач. 
\end{enumerate}